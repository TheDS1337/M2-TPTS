\documentclass[a4paper,11pt]{article}

\usepackage{amsmath}
\usepackage[french]{babel}
\usepackage[T1]{fontenc}
\usepackage[utf8]{inputenc}
\usepackage{lmodern}
\usepackage{microtype}
\usepackage{physics}

\usepackage{hyperref}

\newcommand{\nompropre}[1]{\textit{#1}}
\newcommand{\diff}{\ \mathrm{d}}

\bibliographystyle{unsrt-fr}

\title{Mon premier document \LaTeX \\ Cours de TPTS}
\author{\nompropre{J. Messamah} \and \nompropre{Mazri} \thanks{USTHB}}
\date{}

\begin{document}

\maketitle

\tableofcontents

Voici mon premier document avec \LaTeX

\section*{Introduction générale}
\addcontentsline{toc}{section}{Introduction générale}

Ce cours est basé sur les travaux de la référence \cite{Lamport1994} ainsi que la référence \cite[page 4]{messamah2015quantum}.
\addcontentsline{toc}{section}{Introduction générale}

\section{Introduction}

\section{section 2} %Il ne faut pas mettre de préambule dans le fichier

\section{Test de numérotation}
\label{section/referenceTest}

\section{Commandes de hiérarchisation du document}

Je suis entrain d'écrire et puis je fais référence à la section \ref{section/referenceTest}. On peut aussi utiliser la commande \pageref{section/referenceTest}

\subsection{Ceci est une sous-section}

Ceci est un paragraphe. Dans \LaTeX\ les paragraphes sont séparés par des lignes blanches.

Ceci est un deuxième paragraphe.

\subsubsection{Ceci est une sous-sous-section}

Une note de bas de page \footnote{Ceci est une note de base de page.}.
Une note dans la marge \marginpar{Ceci une une note}

\section{Changement de Style et déclarations}

\textup{droit} , \textit{le texte en italique}, \textbf{texte en gras}.
On peut aussi imbriquer les commandes de changement de style. \textit{\textbf{Ceci est un exemple}}.

{\itshape tout ce que j'écris est maintenant en italique}. Maintenant j'écris normalement.

\section{environnement}

\begin{em}
Le texte contenu dans cet environnement est mis en relief.
\end{em}

\begin{quote}
Voici un exemple de texte sur deux paragraphes, afin de montrer le comportement des environements de citation.

Voici un exemple de texte sur deux paragraphes, afin de montrer le comportement des environements de citation.
\end{quote}

Voici un exemple de l'environement quotation


\begin{quotation}
Voici un exemple de texte sur deux paragraphes, afin de montrer le comportement des environements de citation.

Voici un exemple de texte sur deux paragraphes, afin de montrer le comportement des environements de citation.
\end{quotation}

\begin{center}
Voici un paragraphe centré.
\end{center}

\begin{flushleft}
Voici un paragraphe aligné à gauche.
\end{flushleft}

\begin{flushright}
Voici un paragraphe aligné à droite.
\end{flushright}

Les listes permettent :
\begin{itemize}

\item de structurer ses idées ;

\item d'aérer le texte ;

\item d'améliorer sa lisibilité.

\end{itemize}

\begin{enumerate}

\item premier élément;

\item deuxième élément;

\item troisième élément.

\end{enumerate}

\section{Environnement tabular}

\begin{tabular}{|l|c|r|}
\hline
Sparc & SunOS  & 4.1.4 \\
\hline
Hp     & HP-UX  & 10.20 \\
\hline
PC     & NetBSD & 1.2.1  \\
\hline 
\end{tabular}

\bigskip

\begin{tabular}{|p{5cm}|*{2}{c|}}
\hline
\bfseries Sparc & SunOS  & 4.1.4 \\
\hline
\bfseries Hp     & HP-UX  & 10.20 \\
\hline
\bfseries PC     & NetBSD & 1.2.1  \\
\hline 
\end{tabular}

\bigskip


\begin{tabular}{|p{5cm}|*{2}{c|}}
\hline
					                & Contenance & Quantité \\
\hline
\bfseries Blanche de Bruges & 33 cl            & 10          \\
\hline
\bfseries Guinness               & 1 pint           & 5            \\
\hline
\bfseries Kronenbourg         & 33 cl             & 0            \\
\hline
\end{tabular}

\bigskip

\begin{tabular}{|p{5cm}|*{2}{c|}}
\cline{2-3}
\multicolumn{1}{c|}{}        & \multicolumn{2}{c|}{Commande} \\
\cline{2-3}
\multicolumn{1}{c|}{}        & Contenance & Quantité \\
\hline
\bfseries Blanche de Bruges & 33 cl            & 10          \\
\hline
\bfseries Guinness               & 1 pint           & 5            \\
\hline
\bfseries Kronenbourg         & 33 cl             & 0            \\
\hline
\end{tabular}

\begin{table}[!htbp] % ! respecter l'ordre, h : ici si possible, t : top, b : bottom, p : page of floats
\centering
\begin{tabular}{|l|l|}
\hline
table      & figure   \\
\hline
tableaux & dessins \\
\hline
\end{tabular}
\caption{Exemple d'environnement tableaux}
\label{tableau/exemple}
\end{table}

\section{Mathématique}
Considérons l'équation \begin{math} x + y + z = n \end{math} .
Considéron la variable $ x $. Ou bien \( x\).

Ce qui conduit à :

\begin{displaymath}
x + y + z = n
\end{displaymath}

\[
x+y+z=n
\]

$ x^2 +y^2=1 $ , $ x_1 = x_2 $ , $ x_1^2=x^2_1 $
$ x_{ij} $ , $ x^{y^z} $ , $ x^{yz} $

\[
f(x) > 1 \mbox{ si } x < 3 %\, \: \; \! \quad \qqud pour les espaces
\]

\[
\frac{x}{\sqrt{y}}
\]

\[
\left (\frac{\sqrt[n]{\alpha_i}}{x^2+y^2}\right )
\]
$ \alpha $ , $\beta $ , $\Gamma$ , $\phi$ , $\varphi$

\[
\mathrm{i}\hbar\frac{\mathrm{d}}{\mathrm{d}t} \ket{\psi^\dagger}= \mathbf{H} \ket{\psi}
\]

\[
{\mathcal A} = \left (
\begin{array}{ccc}
a_{11} & a_{12} & a_{13} \\
a_{21} & a_{22} & a_{23} \\
a_{31} & a_{32} & a_{33}
\end{array}
\right )
\]

\[
{\mathcal B} = \left (
\begin{array}{cccc}
a_{11} & a_{12} & \cdots & a_{1n} \\
a_{21} & a_{22} & \cdots & a_{2n} \\
\vdots  & \vdots   & \ddots & \vdots  \\
a_{n1} & a_{n2} & \cdots & a_{nn}
\end{array}
\right )
\]

\begin{equation} \label{equation/euler}
\mathrm{e}^{\mathrm{i}\pi}+1=0
\end{equation}

\begin{eqnarray}
\ln xy                           & = & \ln x + \ln y \\
\exp \left ( x+y \right ) & = & \mathrm{e}^x  \mathrm{e}^y
\end{eqnarray}

\begin{eqnarray}
\int_1^2 x^2 \diff x & = & \left[\frac{x^3}{3}\right ]_1^2 \nonumber \\
                               & = & \frac{7}{3}
\end{eqnarray}

Voir équation \eqref{equation/euler}.

$\vec{a}\cdot\vec{b}$
\appendix
\section{Première annexe}

%\begin{thebibliography}{2}
%\addcontentsline{toc}{section}{Références}
%
%\bibitem{latex-a-document-preparation-system}
%  Leslie Lamport :
%  \textit{\LaTeX: a document preparation system}.
%  Addison-Wesley, second edition, 1994.
%
%\bibitem{the-latex-companion}
%  Frank Mittelbach et Michel Goossens :
%  \textit{The \LaTeX Companion}.
%  Addison-Wesley, second edition, 2004.
%
%\end{thebibliography}

\bibliography{Bibliographie}
\end{document}