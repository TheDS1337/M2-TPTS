\documentclass[a4paper, 11pt]{article}

\usepackage{amsmath, amssymb}
\usepackage[french]{babel}
\usepackage[T1]{fontenc}
\usepackage[utf8]{inputenc}
\usepackage{lmodern}
\usepackage{microtype}
\usepackage{physics}

\usepackage{geometry}
\geometry{a4paper, tmargin = 2cm, bmargin = 2cm, lmargin = 3cm, rmargin = 1cm}
\linespread{1.3}

\usepackage{hyperref}
\hypersetup{colorlinks = true, urlcolor = black, linkcolor = black, filecolor = black, citecolor = blue}

\bibliographystyle{unsrt}

\title{États gaussiens et états cohérents}
\author{\textsc{Mohamed Amine BOUSSOUBEL}}
\date{\today}

\renewcommand{\i}{\mathrm{i}}
\newcommand{\opr}[1]{\hat{#1}}
\newcommand{\diff}{\mathrm{d}}

\begin{document}
	\maketitle
	
	\tableofcontents
	
	\section{Les états gaussiens}
	
	Il existe une classe importante d’états quantiques qui ont la propriété de saturer l’inégalité de Heisenberg. On appelle ces états des \textit{états purs gaussiens}. Dans le cas d’un système à n degrés de liberté, ces états s’écrivent dans la représentation configuration sous la forme générale suivante \cite{Ali2000}
	
	\begin{equation} \label{eq/1}
	\begin{split}
	    \eta_{q, p}^{U, V} 
	        &= \pi^{-\frac{n}{4}} \left[\det U\right] \exp \left[\frac{\i}{2} \left(\mathbf{x} - \frac{\mathbf{q}}{2}\right) \cdot \mathbf{p}\right] \\
	        &\quad \times \exp \left[-\frac{1}{2} \left(\mathbf{x} - \mathbf{q}\right) \cdot \left(U + \i V\right) \left(\mathbf{x} - \mathbf{q}\right)\right],
	 \end{split}
	\end{equation}
	
	\noindent
	où nous avons pris comme unité $\hbar = 1$. Cet état est paramétrépar deux vecteurs à n dimensions $\mathbf{q} = \left(q_1, \ q_2, \ \cdots\!, \ q_n\right), \ \mathbf{p} = \left(p_1, \ p_2, \ \cdots\!, \ p_n\right)$ et deux matrices réelles $U$ et $V$ d’ordre $n \times n$ avec $U$ définie positive. Dans le cas simple où $V$ est nulle, $\mathbf{q} = \mathbf{p} = \mathbf{0}$ et la matrice $U$ est diagonale avec des valeurs propres $\frac{1}{l_i^2}, \ i = 1, \ \cdots\!, \ n$, on retrouve le paquet d’ondes gaussien
	
	\begin{equation} \label{eq/2}
	    \eta^{\mathbf{l}}\left(\mathbf{x}\right) = \prod_{i = 1}^{n} \left(\pi l_i^2\right)^{\frac{1}{4}} \exp \left(- \frac{x_i^2}{2 l_i^2}\right),
	\end{equation}
	
	\noindent
	où le paramètre $\mathbf{l} = (l_1, \ l_2, \ \cdots\!, \ l_n)$ rend compte d’une éventuelle asymétrie dans les fluctuations quantiques des opérateurs position et impulsion dans cet état. En effet, les déviations standards en positions et impulsions dans cet état s’écrivent
	
	\begin{align}
	    {\left<\Delta Q_i\right>}_{\eta^{\mathbf{l}}} &=  \frac{l_i}{\sqrt{2}}, \\
	    {\left<\Delta P_i\right>}_{\eta^{\mathbf{l}}} &=  \frac{1}{\sqrt{2} l_i}.
	\end{align}
	
	Un cas particulier est celui des états à incertitude minimale, avec des fluctuations quantique égales en position et en impulsion. Ceci est réalisé avec des paramètres $l_i = l = 1$ et on a $\left<\Delta Q_i\right> = \left<\Delta P_i\right> = \frac{1}{\sqrt{2}}$. Dans ce cas, l’état quantique \eqref{eq/2} n’est autre que l’état fondamental de l’oscillateur harmonique. Si on parle de quadratures du champ électromagnétique en optique quantique, alors cet état n’est autre que l’état du vide à $n$ modes. Les fluctuations quantiques des quadratures sont appelées quant à elles \textit{bruit standard}.


	
	\section{Les états cohérents}
	Considérons maintenant le cas simple $(n = 1)$. A partir de l’état fondamental de l’oscillateur harmonique, ou l’état du vide, on définit une sous-classe d’états gaussiens (voir la section 1) qui a une grande importance en mécanique quantique : c’est la classe des \textit{états cohérents canoniques} \cite{Glauber1963}. Définissons d’abord les opérateurs de création $\opr{a}^\dagger$ et d’annihilation $\opr{a}$ en fonction des opérateurs position et impulsion

	\begin{align}
	    \opr{a}^\dagger &= \frac{1}{\sqrt{2}} \left(Q - \i P\right), \\
        \opr{a} &= \frac{1}{\sqrt{2}} \left(Q + \i P\right).
	\end{align}
	
	\noindent
	On définit aussi la variable complexe 
	
	\begin{equation} \label{eq/7}
	    z = x + \i y = \frac{1}{\sqrt{2}} \left(q + \i p\right),
	\end{equation}
	
	\noindent
	où $q$ et $p$ sont deux variables réelles. En notant l’état du vide $\eta^{l = 1} \equiv \ket{0}$, l’état cohérent canonique est défini par la relation
	
	\begin{equation} \label{eq/8}
	    \ket{z} = \exp \left(-\frac{\left\lvert z \right\rvert^2}{2} + z \opr{a}^\dagger\right) \ket{0},
	\end{equation}
	
	\noindent
	L’appellation états cohérents provient de l’optique quantique, car on a découvert que le champ électromagnétique dans ces états présentait des propriétés observées avec la lumière cohérente .Lorsqu’un système est dans un état cohérent, l’évolution dans le temps des valeurs moyennes des observables position et impulsion obéit aux lois de la mécanique classique pour l’oscillateur harmonique. Ces états ont les propriétés suivantes :
	
	\begin{enumerate}
	    \item Incertitude minimale
    	    \begin{equation} \label{eq/9}
    	        \left<\Delta Q\right>_z \left<\Delta P\right>_z = \frac{1}{2};
    	    \end{equation}
    	    
        \item États propres de l’opérateur annihilation
            \begin{equation} \label{eq/10}
                \opr{a} \ket{z} = z \ket{z};
            \end{equation}
            
        \item Générés à partir de l’état du vide $\ket{0}$ par l’action du groupe de Weyl-Heisenberg
            \begin{equation} \label{eq/11}
                \ket{z} = \mathrm{D}(z) \ket{0},
            \end{equation}
            
            \noindent
            où l’opérateur $\mathrm{D}(z)$ est l’opérateur déplacement dans l’espace des phases
	        \begin{equation} \label{eq/12}
                \mathrm{D}(z) = \exp \left(z \opr{a}^\dagger - \overline{z} \opr{a}\right);
            \end{equation}
            
        \item Constituent une famille sur-complète d’états
            \begin{equation} \label{eq/13}
                \frac{1}{\pi} \int\limits_{\mathbb{C}} \ket{z}\!\bra{z} \diff\!\Re\!z\diff\!\Im\!z = \mathbf{I},
            \end{equation}
            
            \noindent
            où $\mathbf{I}$ est l’opérateur identité.
	    
	\end{enumerate}

    \addcontentsline{toc}{section}{Références}
    \bibliography{References}

\end{document}